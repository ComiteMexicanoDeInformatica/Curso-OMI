\documentclass[10pt,letterpaper,twocolumn,spanish]{article}
\usepackage{babel}
\usepackage[margin=8mm]{geometry}
\usepackage[cp1252]{inputenc}
\usepackage{graphicx}
\usepackage{pstricks}
\usepackage{amsfonts}
\usepackage{paralist}

\newpsobject{showgrid}{psgrid}{subgriddiv=1,griddots=10,gridlabels=0}

\title{\vspace{-10mm}
Introducción a las estructuras de datos
}
\author{Workshop COMI}
\date{\today}

\pagestyle{empty}

\begin{document}

\maketitle
\thispagestyle{empty}

\section*{Nivel IPO}

En este primer nivel (\textit{Inputs, Process, Outputs}) ya deberías poder resolver retos interactivos elementales que reciben entradas, realizan procesos simples (fórmulas) y emiten salidas. Esto implica una buena comprensión de los siguientes conceptos:

\begin{compactitem}
\item Tipos de datos
\begin{compactitem}
\item Primitivos
\begin{compactitem}
\item Numéricos
\begin{compactitem}
\item Enteros
\item Reales
\end{compactitem}
\item Alfanuméricos
\begin{compactitem}
\item Caracteres
\item Cadenas
\end{compactitem}
\item Booleanos
\end{compactitem}
\item Derivados
\begin{compactitem}
\item Arreglos
\item Referencias
\item Apuntadores
\end{compactitem}
\end{compactitem}
\item Expresiones (precedencia y asociatividad)
\item Sentencias
\end{compactitem}

\subsubsection*{Enunciado 1}

Dado un entero positivo $n$, calcula la suma de todos los enteros desde 1 hasta $n$.

\subsubsection*{Análisis 1}

Puedes aprovechar la suma de Gauss para hacer un código elemental de tipo interactivo.
\[1+2+3+\cdots+(n-1)+n=\sum_{i=1}^{n}i=\frac{n(n+1)}{2}\]

Realizar este programa con un bucle \textit{no} es una buena idea:

\begin{verbatim}
  unsigned int sum = 0;
  for(unsigned int i = 1; i <= n; ++i) {
    sum += i;
  }
\end{verbatim}

El bucle es más complejo (costoso) que la fómula directa.

\subsubsection*{Solución 1}

\begin{verbatim}
#include<iostream>
int main() {
  using namespace std;
  cout << "Valor de N: ";
  unsigned int n;
  cin >> n;
  unsigned int sum = n * (n + 1) / 2;
  cout << "La suma es " << sum << '\n';
  return 0;
}
\end{verbatim}

\section*{Nivel SCIB}

El segundo nivel abarca las estructuras de control de flujo, con las que aumenta mucho tu poder de cómputo.

\begin{compactitem}
\item Secuencia
\item Condición
\item Iteración
\item Bifurcación
\end{compactitem}

\subsubsection*{Enunciado 2}

Para cada entero positivo $n$ de la entrada, imprime el número de ternas pitagóricas primitivas cuyos lados no excedan $n$.

\subsubsection*{Análisis 2}

Las siguientes fórmulas producen ternas pitagóricas para cualesquier $m > n \in \mathbb{Z}^+$; la terna producida será primitiva si y sólo si $m$ y $n$ son primos relativos de distinta paridad.

\begin{eqnarray*}
a & = & m^2 - n^2 \\
b & = & 2mn \\
c & = & m^2 + n^2 \\
\end{eqnarray*}

\begin{compactitem}
\item
El uso del \texttt{typedef} al inicio del código facilita el escalamiento de los tipos de datos.
\item
El cálculo del máximo común divisor (\texttt{gcd}) es ideal para expresarlo en una función aparte.
\item
Formalmente, \texttt{goto} no es indispensable porque el cálculo de las ternas puede enviarse a otra función; con esto \texttt{main} quedaría elementalmente simple:
\begin{verbatim}
int main() {
  integer max;
  while(cin >> max) {
    cout << ppt(max) << '\n';
  }
  return 0;
}
\end{verbatim}
Sin embargo, el ejemplo ilustra cómo aprovecharlo para salir selectivamente de una profunda anidación de iteraciones.
\item
El inicializador de \texttt{n} (en el segundo \texttt{for}) ilustra cómo \textit{construir} un valor que sea de distinta paridad con respecto a \texttt{m}; esto es menos complejo que correr unitariamente a \texttt{n} y preguntar si sus módulos 2 son distintos.
\item
El primer valor de \texttt{c} que pase de \texttt{max} no es necesariamente el fin del proceso; cuando esto suceda en el \textit{primer} valor de \texttt{n} para un cierto \texttt{m}, entonces ya podremos estar seguros de que no hay más ternas cuya hipotenusa no exceda de \texttt{max}.
\item
He aquí la versión recursiva el máximo común divisor; sin embargo, la recursión suele ser más valiosa cuando la implementación iterativa es muy ardua.

\begin{verbatim}
integer gcd(integer m, integer n) {
  return !b ? a : gcd(b, a % b);
}
\end{verbatim}
\end{compactitem}

\subsubsection*{Solución 2}
\begin{verbatim}
#include<iostream>
#include<algorithm>

using namespace std;

typedef unsigned long long int integer;

integer gcd(integer m, integer n) {
  while(m) {
    if(m < n) {
      swap(m, n);
    }
    m %= n;
  }
  return n;
}

int main() {
  integer max;
  while(cin >> max) {
    integer count = 0;
    for(integer m = 2; m <= max; ++m) {
      integer mm = m * m;
      for(integer n = 1 + (m % 2); n <= m; n += 2) {
        if(gcd(m, n) == 1) {
        integer nn = n * n;
        integer a = mm - nn;
        integer b = 2 * m * n;
        integer c = mm + nn;
        if(c <= max) {
          if((a * a + b * b) == (c * c)) {
            ++count;
          }
        } else if(n < 3) {
          goto next;
        }
      }
    }
    next:
    cout << count << '\n';
  }
  return 0;
}
\end{verbatim}

\section*{Nivel ED}

El tercer nivel implica el uso de estructuras de datos, las cuales son máquinas que procesan secuencias de datos simples a través de diversos mecanismos de acceso, recuperación, inserción o eliminación;
dominar este nivel ya te deja en posición de competir en las fases iniciales de la Olimpiada de Informática.

Hay 2 máquinas básicas (arreglos y listas) a partir de las cuales se construyen algunas derivadas (vectores, pilas, colas, y árboles).

\subsection*{Arreglos}

Son la estructura de datos más simple:
\begin{compactitem}
\item Datos del mismo tipo y físicamente contiguos en RAM
\item Acceso aleatorio (por índice) en tiempo constante
\item Inserciones y eliminaciones costosas
\end{compactitem}

\begin{center}
\psset{unit=4mm}
\begin{pspicture}(0,0)(24,6)
%\showgrid
\rput[t](12,6){\texttt{float a[3];}}
\rput[t](4,6){¡ilegal!}
\rput[t](20,6){¡ilegal!}
\rput[t](12,5){$\overbrace{\textcolor{white}{.\hspace{45mm}.}}$}
\rput[t](4,5){$\downarrow$}
\rput[t](20,5){$\downarrow$}
\rput(4,3.5){\texttt{a[-1]}}
\rput(8,3.5){\texttt{a[0]}}
\rput(12,3.5){\texttt{a[1]}}
\rput(16,3.5){\texttt{a[2]}}
\rput(20,3.5){\texttt{a[3]}}
\psset{linewidth=0.75mm}
\psframe(2,2)(6,3)
\psframe(6,2)(10,3)
\psframe(10,2)(14,3)
\psframe(14,2)(18,3)
\psframe(18,2)(22,3)
\psset{linewidth=0.1mm}
\psframe(3,2)(4,3) \psframe(4,2)(5,3)
\psframe(7,2)(8,3) \psframe(8,2)(9,3)
\psframe(11,2)(12,3) \psframe(12,2)(13,3)
\psframe(15,2)(16,3) \psframe(16,2)(17,3)
\psframe(19,2)(20,3) \psframe(20,2)(21,3)
\rput[t](6.5,1.75){$\uparrow$}
\rput[b](6.5,0){\texttt{a} $\equiv$ \texttt{\&a[0]}}
\rput[t](18.5,1.75){$\uparrow$}
\rput[b](18.5,0){\texttt{a+3} $\equiv$ \texttt{\&a[3]}}
\end{pspicture}
\end{center}

\begin{compactitem}
\item
Un arreglo estático de $n$ variables (en la imagen, son tres \texttt{float}) hace que el compilador reserve espacio contiguo en RAM suficiente (como \texttt{sizeof(float)} es 4, esto equivale a 12 bytes en el ejemplo).
\item
A partir de esto, \texttt{a[0]}, \texttt{a[1]}, \ldots, \texttt{a[n-1]} son variables simples; en este diagrama, el último valor es \texttt{a[2]}.
\item
Aunque técnicamente la expresión \texttt{a[-1]} es legal (se trata de los 4 bytes inmediatamente antecesores a \texttt{a[0]} vistos como \texttt{float}), es completamente ilegal acceder de cualquier forma a esa variable, ya que la RAM que le corresponde no ha sido gestionada y se obtendría una infracción de acceso.
\item
La expresión \texttt{a[n]} es legal y corresponde a los 4 bytes inmediatamente posteriores al último elemento del arreglo, vistos como \texttt{float}. No está permitido cambiar el estado de esa variable (nuevamente se tendría una infracción de acceso) pero su dirección (\texttt{a+n}) es legal y resulta útil como el extremo final exclusive para iterar eficientemente sobre el arreglo.
\end{compactitem}

He aquí cómo adquirir mediante un índice entero los datos de un arreglo estático:

\begin{verbatim}
const int N = 10;
float a[N];
for(int i = 0; i < N; ++i) {
  cin >> a[i];
}
\end{verbatim}

He aquí cómo recorrer eficientemente el arreglo:

\begin{verbatim}
float *i = a;
float *end = a + N;
while(i < end) {
  cout << *i++ << '\n';
}
\end{verbatim}

Un gran inconveniente de los arreglos es el hecho de que agrandarlos implica pedir dinámicamente un nuevo bloque de RAM y copiar los datos del original en el nuevo. Eso es lo que hace un arreglo dinámico como \texttt{vector}:

\begin{verbatim}
vector< float > v;
float f;
while(cin >> f) {
  v.push_back(f);
}
\end{verbatim}

Sin embargo, el recorido sigue siendo análogamente muy eficiente:

\begin{verbatim}
vector< float >::const_iterator i = v.begin();
while(i != v.end()) {
  cout << *i++ << '\n';
}
\end{verbatim}

La mayor utilidad de los arreglos está en el manejo de información multidiminsionada, como las matrices.

\subsubsection*{Enunciado 3}

Tu entrada es una matriz binaria (de 0's y 1's) cuyo tamaño se indica con dos enteros $r$ y $c$ (renglones y columnas, respectivamente, ambos entre 1 y 100) seguidos de los bits. Deberás contar cuántas cruces existen en la matriz; una cruz está formada por cinco 1's colocados contiguamente. Por ejemplo, ante la siguiente entrada tu programa debería contar 3.
\begin{verbatim}
4 7
0111000
1111010
0110111
0000010
\end{verbatim}

\subsubsection*{Análisis 3}

Un arreglo bidimensional facilita el acceso arbitrario a cada dato; nota cómo los ciclos de búsqueda trabajan en $[1,n-1]$; esto facilita el conteo porque evita tomar consideraciones especiales en los bordes.

\subsubsection*{Solución 3}

\begin{verbatim}
#include<iostream>

int main() {
  using namespace std;
  const int MAX = 100;
  int a[MAX][MAX] = { 0 };
  int max_r, max_c, n, r, c;
  char ch;
  if(!(cin >> max_r >> max_c)) {
    return 1;
  }
  if((max_r < 1) || (max_r > MAX)) {
    return 2;
  }
  if((max_c < 1) || (max_c > MAX)) {
    return 3;
  }
  for(r = 0; r < max_r; ++r) {
    for(c = 0; c < max_c; ++c) {
      if(!(cin >> ch)) {
        return 4;
      }
      a[r][c] = (ch == '1');
    }
  }
  int sum, count = 0;
  for(r = 1; r < max_r - 1; ++r) {
    for(c = 1; c < max_c - 1; ++c) {
      sum
        = a[r][c] + a[r+1][c] + a[r-1][c]
        + a[r][c+1] + a[r][c-1]
      ;
      if(sum == 5) {
        ++count;
      }
    }
  }
  cout << count << '\n';
  return 0;
}
\end{verbatim}

\subsection*{Listas}

Una lista ligada computacional es una secuencia de nodos (ubicados de forma arbitraria en RAM) cada uno de los cuales contiene una pieza de infomación (dato, los cuadrados en el diagrama) y una o más conexiones hacia otros nodos (los triángulos). Se requiere al menos un punto de acceso global o enlace al primer dato (que puede ser \textit{tierra} si la lista está vacía); ése enlace simple (no nodo) aparece en el extremo izquierdo.

\begin{center}
\psset{unit=4mm}
\begin{pspicture}(0,4)(21,7)
\rput[lb](0,5){\psline(0,0)(0,2)(2,2)(0,0)\rput(0.5,1.5){$\bullet$}\rput(0.5,1.5){\psline{->}(0,0)(2.5,0)}}
\rput[lb](3,4){\psframe(0,0)(3,3)}
\rput[lb](6,5){\psline(0,0)(0,2)(2,2)(0,0)\rput(0.5,1.5){$\bullet$}\rput(0.5,1.5){\psline{->}(0,0)(2.5,0)}}
\rput[lb](9,4){\psframe(0,0)(3,3)}
\rput[lb](12,5){\psline(0,0)(0,2)(2,2)(0,0)\rput(0.5,1.5){$\bullet$}\rput(0.5,1.5){\psline{->}(0,0)(2.5,0)}}
\rput[lb](15,4){\psframe(0,0)(3,3)}
\rput[lb](18,5){\psline(0,0)(0,2)(2,2)(0,0)\rput(0.5,1.5){$\bullet$}\rput(0.5,1.5){\psline(0,0)(2,0)(2,-0.5)}}
\rput(20,6){
\psline(0,0)(1,0)
\psline(0.1,-0.2)(0.9,-0.2)
\psline(0.2,-0.4)(0.8,-0.4)
\psline(0.3,-0.6)(0.7,-0.6)
\psline(0.4,-0.8)(0.6,-0.8)
}
%\showgrid
\end{pspicture}
\end{center}

Las listas son ideales para el proceso secuencial de los datos y para inserciones y eliminaciones arbitarias; en cambio, son linealmente costosas para el acceso arbitrario del tipo arreglo.

Pueden ser simples o dobles; éstas contienen dos ligas (al anterior y al siguiente), lo que permite el acceso bidireccional.

Pueden ser lineales o circulares (cuando el último nodo enlaza al primero), lo facilita el acceso cíclico a una secuencia de datos.

\subsubsection*{Enunciado 4}

Dada una secuencia de reales imprimir en la salida su desviación estándar.

\subsubsection*{Análisis 4}

La desviación estándar discreta se define como la raíz cuadrada de la varianza de la distribución de probabilidad discreta:

\[s = \sqrt{\frac{\sum_{i=1}^{n}(x_i - \overline{x})^2}{n}}\]

Un cálculo que sólo requiere una pasada (la suma, la media) puede hacerse \textit{in situ}, por lo que ni siquiera requiere arreglos:

\begin{verbatim}
  int count = 0;
  double f, sum = 0;
  while(cin >> f) {
    sum += f;
    ++count;
  }
  if(count) {
    cout << (sum / count) << '\n';
  }
\end{verbatim}

En cambio, la desviación estándar requiere \textit{dos} pasadas: una para calcular la media y otra para la acumulación de los cuadrados de las diferencias.

Un lista simple es ideal para este reto porque permite adquirir los datos en una primera pasada (al mismo tiempo, calcular la media) y recorrer de nuevo los datos para el cálculo final.

Podría calcularse el tamaño actual de la lista así:

\begin{verbatim}
  int sz = lst.size();
\end{verbatim}

Pero esto puede costar (en algunas implementaciones) otra pasada; es menos complejo contar desde la adquisición.

\subsubsection*{Solución 4}

\begin{verbatim}
#include<iostream>
#include<cmath>
#include<list>

int main() {
  using namespace std;
  typedef long double real;
  list< real > lst;
  real r, sum = 0;
  int count = 0;
  while(cin >> r) {
    lst.push_back(r);
    sum += r;
    ++count;
  }
  if(count) {
    real mean = sum / count;
    sum = 0;
    list< real >::const_iterator i = lst.begin();
    while(i != lst.end()) {
      r = *i++ - mean;
      sum += r * r;
    }
    cout << sqrt(sum / count) << '\n';
  }
  return 0;
}
\end{verbatim}

\[\diamond\]

A partir de estas dos estructuras de datos elementales podemos modelar máquinas de proceso sumamente útiles para el desarrollo de nuestros algoritmos.

\subsection*{Vectores}

Un vector es una generalización del concepto de arreglo. No obstante, puede implementarse mediante un arreglo dinámico o mediante una lista ligada; la elección depende del tipo de acceso (secuencial o arbitrario) que necesitemos.

Por ejemplo, puedes reemplazar todas las ocurrencias de la palabra \texttt{list} por la palabra \texttt{vector} en el código anterior sin que su funcionamiento se vea alterado. Sin embargo, el proceso de lectura de los datos perderá eficiencia cada vez que el vector alcance el tope de su capacidad porque antes de poder insertar un nuevo dato será necesario un proceso de reubicación (\textit{reallocating}).

Analiza el siguiente código en tu sistema local para que comprendas cómo va reubicándose el arreglo conforme va creciendo; en mi compilador se reubica los primeros 5 datos y de ahí la capacidad salta así: 6, 9, 13, 19, 28, 42, 63, 94, 141, 211, 316, 474, 711, 1066, etcétera.

\begin{verbatim}
#include<iostream>
#include<vector>

int main() {
  using namespace std;
  vector< int > v;
  // v.reserve(1000);
  cout
    << '\t'                 // aún no hay datos
    << v.size() << '\t'     // vector vacío
    << v.capacity() << '\n' // depende del implementador
  ;
  int i;
  while(cin >> i) {
    v.push_back(i);
    cout
      << i << '\t'            // dato
      << v.size() << '\t'     // tamaño actual
      << v.capacity() << '\n' // cómo y cuándo crece
    ;
  }
  return 0;
}
\end{verbatim}

Para evitar esto puedes descomentar la invocación al método \texttt{reserve}; esto garantiza que no habrá reubicación antes de las primeras 1000 inserciones en nuestro ejemplo.

Los vectores basados en arreglos dinámicos pueden eliminar eficientemente a su último elemento con el método \texttt{pop\_back} (siempre y cuando no esté vacío el vector).

Por último, los vectores hacen simple (aunque no muy eficiente) la gestión de matrices dinámicas como la del enunciado 3:

\begin{verbatim}
vector< vector < int > > a(
  max_r,
  vector< int >(max_r, 0)
);
\end{verbatim}

Con esta declaración, \texttt{a} es una matriz dinámica (del tamaño exacto) e inicializada con ceros.

\subsubsection*{Enunciado 5}

La entrada es un entero que anuncia el número de pares de flotantes que vienen a continuación y que representan dos vectores paralelos; tu programa deberá calcular e imprimir el producto punto inverso; es decir, la suma de los productos del primero del vector izquierdo con el último del vector derecho, el segundo del vector izquierdo con el penúltimo del vector derecho; y así sucesivamente hasta el último de vector izquierdo con el primero del vector derecho. Por ejemplo, con la entrada siguiente el resultado será 0.6:

\begin{verbatim}
4
0.1 0.2
0.3 0.4
0.5 0.6
0.7 0.8
\end{verbatim}

\subsubsection*{Análisis 5}

Como se nos da el tamaño por anticipado, podemos reservar el tamaño de dos vectores paralelos para reducir a cero las reubicaciones; el proceso no puede hacerse en una sola pasada porque el vector derecho debe tomarse en orden inverso.

Nota el uso de los iteradores directos (\texttt{const\_iterator}) y reversos(\texttt{const\_reverse\_iterator}) para recorrer eficientemente los vectores; no se usa el decremento sobre el iterador reverso porque prevalece la noción de \textit{avanzar} sobre el vector.

\subsubsection*{Solución 5}

\begin{verbatim}
#include<iostream>
#include<vector>

int main() {
  using namespace std;
  int n;
  if(cin >> n) {
    vector< float > v1, v2;
    v1.reserve(n);
    v2.reserve(n);
    float f1, f2;
    while(n-- && (cin >> f1 >> f2)){
      v1.push_back(f1);
      v2.push_back(f2);
    }
    float sum = 0;
    vector< float >::const_iterator i = v1.begin();
    vector< float >
      ::const_reverse_iterator j = v2.rbegin();
    while(i != v1.end()) {
      sum += *i++ * *j++;
    }
    cout << sum << '\n';
  }
  return 0;
}
\end{verbatim}

\subsection*{Pilas}

Una pila es una máquina de datos gobernada por el mecanismo LIFO (\textit{last in, first out}). Los cuatro métodos relevantes para su gestión son:

\begin{compactitem}
\item \texttt{empty} (estado inicial)
\item \texttt{push} (inserción)
\item \texttt{pop} (extracción)
\item \texttt{top} (copia del valor superior)
\end{compactitem}

No se necesita acceso a ningún dato que no sea el tope.

Las pilas pueden ser eficientemente implementadas con lista ligadas simples (en las que el enlace principal apunta al tope) o con arreglos (cuando puede preverse una capacidad máxima). Su propia naturaleza las hace poco propensas a problemas de fragmentación de memoria.

\subsubsection*{Enunciado 6}

Dada una secuencia de palabras, imprime en orden inverso de llegada aquéllas que inicien con mayúscula; ante la entrada \texttt{Alberto y Beatriz fueron a la playa con Carmen} tu programa debería imprimir \texttt{Carmen Beatriz Alberto}.

\subsubsection*{Análisis 6}

Una pila es la herramienta ideal; sólo falta filtrar lo que se le pone en ella.

\subsubsection*{Solución 6}

\begin{verbatim}
#include<iostream>
#include<locale>
#include<string>
#include<stack>

int main() {
  using namespace std;
  stack< string > s;
  string w;
  while(cin >> w) {
    if(isupper(w[0])) {
      s.push(w);
    }
  }
  while(!s.empty()) {
    cout << s.top() << '\n';
    s.pop();
  }
  return 0;
}
\end{verbatim}

\subsection*{Colas}

Al contrario que las pilas, el mecanismo clave en las colas es FIFO (\textit{first in, first out}). Pueden ser implementadas mediante listas ligadas simples o dobles (según el tipo de recorrido que se precise hacer sobre sus datos).

Los métodos disponibles son los siguientes:

\begin{compactitem}
\item \texttt{empty} (estado inicial)
\item \texttt{size} (tamaño actual)
\item \texttt{front} (cabeza)
\item \texttt{back} (cola)
\item \texttt{push} (insersión por la cola)
\item \texttt{pop} (extracción por la cabeza)
\end{compactitem}

El uso de arreglos dinámicos para construir colas eficientes queda restringido a los casos en que el total de datos encolados no rebase la capacidad del vector; sin embargo, cuando se dispone del tamaño límite, los arreglos son muy eficientes para gestionar colas circulares.

\subsubsection*{Enunciado 7}

Tu programa recibirá una lista de números reales y deberá enviarlos a la salida aumentados en el absoluto del primer valor negativo que aparezca en la misma secuencia. Ante una entrada como \texttt{1 2 -3 4 5 6} tu programa debería emitir \texttt{4 5 0 7 8 9}.

\subsubsection*{Análisis 7}

Podríamos leer todos los datos en un \texttt{vector} y volverlo a recorrer para aplicar el desplazamiento numérico, pero el uso de \texttt{queue} es más natural y acorde al proceso que se implementa.

\subsubsection*{Solución 7}

\begin{verbatim}
#include<iostream>

int main() {
  using namespace std;
  queue< float > q;
  float f, offset;
  while(cin >> f) {
    q.push(f);
    if(f < 0) {
      offset = f;
      break;
    }
  }
  while(cin >> f) {
    q.push(f);
  }
  while(!q.empty()) {
    cout << (q.front() - offset) << '\n';
    q.pop();
  }
  return 0;
}
\end{verbatim}

\subsection*{Árboles}

Un árbol es un tipo especial de lista doblemente enlazada que contiene enlaces hacia la izquierda (datos menores que el actual) o hacia la derecha (datos mayores que el actual). Esta estuctura es muy útil para implementar los mapas o arreglos asociativos, en lo que el índice no es necesariamente un entero en [1,n].

\subsubsection*{Enunciado 8}

Realiza un programa que presente un histograma de las frecuencias de las palabras de su entrada, ordenadas alfabéticamente.

\subsubsection*{Análisis 8}

Podemos leer las cadenas en una estructura lineal, ordenarlas y luego contarlas, pero es mucho más claro (y, en función del exacto desorden de los datos, generalmente más eficiente) hacer uso de un arreglo asociativo.

\subsubsection*{Solución 8}

\begin{verbatim}
#include<iostream>
#include<string>
#include<map>

int main() {
  using namespace std;
  string w;
  map< string, int > m;
  while(cin >> w) {
    ++m[w];
  }
  map< string, int >::const_iterator i = m.begin();
  while(i != m.end()) {
    cout << i->first << '\t' << i->second <<'\n';
    ++i;
  }
  return 0;
}
\end{verbatim}

\[\diamond\]

Aquí hemos presentado una introducción a las principales estructuras de datos que te servirán para desarrollar tus algoritmos ante los retos de la olimpiada.

\end{document}
